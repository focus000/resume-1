% !TEX program = xelatex

\documentclass{resume}
%\usepackage{zh_CN-Adobefonts_external} % Simplified Chinese Support using external fonts (./fonts/zh_CN-Adobe/)
%\usepackage{zh_CN-Adobefonts_internal} % Simplified Chinese Support using system fonts

\begin{document}
\pagenumbering{gobble} % suppress displaying page number

\name{Yunfang Li}

\basicInfo{
  \email{liyunfang@outlook.com} \textperiodcentered\ 
  \phone{(+86) 130-0877-2794} \textperiodcentered\ 
  % \github[@focus000]{https://github.com/focus000} \textperiodcentered\ 
  \linkedin[Yunfang-Li]{https://www.linkedin.com/in/yunfang-li-3ba402ba/}
  }

\section{\faGraduationCap\ Education}
\datedsubsection{\textbf{Lanzhou University (LZU)}, Gansu, China}{2017 -- Present}
\textit{Master student (Advisor: Prof. Shan Ma)} in Applied Mathematics, GPA: 3.32/4\\
The Third Prize Scholarship (2017, 2018 and 2019), expected July 2020
\datedsubsection{\textbf{Huaibei Normal University}, Anhui, China}{2013 -- 2017}
\textit{B.S.} in Applied Mathematics, GPA: 3.1/4\\
Third Prize of Contemporary Undergraduate Mathematical Contest in Modeling

% \section{\faFlask\ Research Experience}
% \subsection{\textbf{Dynamic System - Long Time Behavior of Evolution Equations}}
% \begin{itemize}
%   \item Give the existence of attractors of weighted $p$-Laplacian in case to portray long time behavior of solutions
%   \item Estimate the regularity of global attractors which is helpful to research further property, such as dimension
%   \item The main frameworks of research are measure theory, nonlinear functional analysis and PDE's prior estimate
% \end{itemize}

% \subsection{\textbf{Machine Learning - Machine Learning via Dynamical Systems}}
% \begin{itemize}
%   \item Study machine learning by dynamic systems, which is considering the neural network as a continuous dynamic system via turning the layer index into time
%   \item From this perspective, we could design a new network by PDE/ODE models
%   \item We could solve optimization problems via training networks
% \end{itemize}

\section{\faBriefcase\ Experience}
\datedsubsection{\textbf{Software Engineer at Huawei}}{2020.8 -- now}
\textit{develop audio engine in a video meeting sdk}
\begin{itemize}
  \item Implemented feature about sharing computer sound during screen sharing in Windows, defect density < 0.7 pcs/kloc. Use Windows WASAPI implemented
  loopback recording and stream routing. Adapted 3A algorithm and took audio mix for the audio stream. 
  \item Optimized audio effect. Use Hidden Markov Model identifed various usage scenes and modified algorithm for specific scene,
  then solved the problem of echo leakage and over-suppression in some headset scenes. It has solved the problem of echo leakage in speaker mode by using
  a series of methods such as dynamic gain control rendering strean and adujust rendering data delay.
  Through modified some parameters in 3A algorithm, the audio effect in double talk scene has been greatly improved.
  \item Implemented DNN denoise and Howling detect features as highlights of this product, improved product competitiveness.
  \item Give a talk about design and the principle of the 3A algorithm. Summarized the algorithm code architecture.
\end{itemize}
\textit{develop audio engine in IdeaHub}
\begin{itemize}
  \item Analyzed audio over-suppression problem and gave a circumvention plan.
  \item Analysed the audio effect of third party meeting apps running on IdeaHub and gave a instruction manual about how to config IdeaHub can improve audio effect of third party meeting apps.
  \item Designed and implemented queue cleaning mechanism, solved the echo leakage problem with high CPU usage.
\end{itemize}
% \datedsubsection{\textbf{Software Interface Development}}{Dec. 2019 -- Present}
% \role{Python, Qt5}{collaborated with The Institute of Modern Physics (IMP) of CAS}
% Use PyQt5 to develop a GUI with is control heavy ion accelerator
% \begin{itemize}
%   \item Organize work and write project documentation.
%   \item Responsible for the architecture design of the client
%   \item Complete the desktop application development and optimization of existing products on time according to product requirements and project mission plans
% \end{itemize}

% Reference Test
%\datedsubsection{\textbf{Paper Title\cite{zaharia2012resilient}}}{May. 2015}
%An xxx optimized for xxx\cite{verma2015large}
%\begin{itemize}
%  \item main contribution
%\end{itemize}

\section{\faCogs\ Skills}
\begin{itemize}
  \item Programming Languages: C, C++, Python
  \item Frameworks\&Tools: PyQt, Git, CMake, Windbg
  \item Languages: English - Fluent(CET6), Mandarin - Native speaker
  \item Understand the implementation of WebRTC device management and the basic principles of AEC3 algorithm
  \item Understand the implementation of VLC audio capture and playback
  \item Proficiency in Debug and performance analysis using Visual Studio and XCode
\end{itemize}
% \section{\faInfo\ Miscellaneous}
% \datedsubsection{International Conference on Infinite-Damensional Dynamical Systems and Attractors}{2019.7}
% \datedsubsection{Summer School, \textit{Lanzhou University}}{2019.6}
% \datedsubsection{Summer School, \textit{Xiamen University}}{2018.8}
% \datedsubsection{International Conference on Nonlinear Analysis and 20th National Conference on Nonlinear Functional Analysis}{2018.5}
\section{\faGamepad\ About Me}
I am looking for a software engineer position, more inclined to back-end and low-level software development, looking forward to the work location in the Yangtze River Delta region.
%% Reference
%\newpage
%\bibliographystyle{IEEETran}
%\bibliography{mycite}
\end{document}
