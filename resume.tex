% !TEX program = xelatex

\documentclass{resume}
%\usepackage{zh_CN-Adobefonts_external} % Simplified Chinese Support using external fonts (./fonts/zh_CN-Adobe/)
%\usepackage{zh_CN-Adobefonts_internal} % Simplified Chinese Support using system fonts

\begin{document}
\pagenumbering{gobble} % suppress displaying page number

\name{Yunfang Li}

\basicInfo{
  \email{liyunfang@outlook.com} \textperiodcentered\ 
  \phone{(+86) 137-0877-2794} \textperiodcentered\ 
  \github[@focus000]{https://github.com/focus000} \textperiodcentered\ 
  \linkedin[Yunfang-Li]{https://www.linkedin.com/in/yunfang-li-3ba402ba/}}

\section{\faGraduationCap\ Education}
\datedsubsection{\textbf{Lanzhou University (LZU)}, Gansu, China}{2017 -- Present}
\textit{Master student (Advisor: Prof. Shan Ma)} in Applied Mathematics, GPA: 3.32/4\\
The Third Prize Scholarship (2017, 2018 and 2019), expected July 2020
\datedsubsection{\textbf{Huaibei Normal University}, Anhui, China}{2013 -- 2017}
\textit{B.S.} in Applied Mathematics, GPA: 3.1/4

\section{\faFlask\ Research Experience}
\subsection{\textbf{Dynamic System - Long Time Behavior of Evolution Equations}}
\begin{itemize}
  \item Give the existence of attractors of weighted $p$-Laplacian in case to portray long time behavior of solutions
  \item Estimate the regularity of global attractors which is helpful to research further property, such as dimension
  \item The main frameworks of research are measure theory, nonlinear functional analysis and PDE's prior estimate
\end{itemize}

\subsection{\textbf{Machine Learning - Machine Learning via Dynamical Systems}}
\begin{itemize}
  \item Study machine learning by dynamic systems, which is considering the neural network as a continuous dynamic system via turning the layer index into time
  \item From this perspective, we could design a new network by PDE/ODE models
  \item We could solve optimization problems via training networks
\end{itemize}

\section{\faUsers\ Experience}
\datedsubsection{\textbf{Software Interface Development}}{Dec. 2019 -- Present}
\role{Python, Qt5}{collaborated with The Institute of Modern Physics (IMP) of CAS}
Use PyQt5 to develop a GUI with is control heavy ion accelerator
\begin{itemize}
  \item Organize work and write project documentation.
  \item Use QWidget to write most of the interface, consider refactoring via QML.
  \item Generate json file by using regex match, and then generate the code automatically.
  \item Use PyQtGraph to plot graphs.
\end{itemize}

% Reference Test
%\datedsubsection{\textbf{Paper Title\cite{zaharia2012resilient}}}{May. 2015}
%An xxx optimized for xxx\cite{verma2015large}
%\begin{itemize}
%  \item main contribution
%\end{itemize}

\section{\faCogs\ Skills}
\begin{itemize}[parsep=0.5ex]
  \item Programming Languages: C++, Python, Matlab
  \item Platform: Linux, macOS
  \item Frameworks: Qt5, Pytorch, Tensorflow
  \item Languages: English - Fluent(CET6), Mandarin - Native speaker
\end{itemize}

\section{\faInfo\ Miscellaneous}
\datedsubsection{International Conference on Infinite-Damensional Dynamical Systems and Attractors}{2019.7}
\datedsubsection{Summer School, \textit{Lanzhou University}}{2019.6}
\datedsubsection{Summer School, \textit{Xiamen University}}{2018.8}
\datedsubsection{International Conference on Nonlinear Analysis and 20th National Conference on Nonlinear Functional Analysis}{2018.5}

%% Reference
%\newpage
%\bibliographystyle{IEEETran}
%\bibliography{mycite}
\end{document}
