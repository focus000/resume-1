% !TEX TS-program = xelatex
% !TEX encoding = UTF-8 Unicode
% !Mode:: "TeX:UTF-8"

\documentclass{resume}
\usepackage{zh_CN-Adobefonts_external} % Simplified Chinese Support using external fonts (./fonts/zh_CN-Adobe/)
% \usepackage{NotoSansSC_external}
% \usepackage{NotoSerifCJKsc_external}
% \usepackage{zh_CN-Adobefonts_internal} % Simplified Chinese Support using system fonts
\usepackage{linespacing_fix} % disable extra space before next section
\usepackage{cite}

\begin{document}
\pagenumbering{gobble} % suppress displaying page number

\name{李蕴方}

\basicInfo{
  \email{liyunfang@outlook.com} \textperiodcentered\ 
  \phone{(+86) 137-0877-2794} \textperiodcentered\ 
  \github[@focus000]{https://github.com/focus000} \textperiodcentered\ 
  \linkedin[Yunfang-Li]{https://www.linkedin.com/in/yunfang-li-3ba402ba/}}
 
\section{\faGraduationCap\  教育背景}
\datedsubsection{\textbf{兰州大学}, 兰州, 甘肃}{2017 -- 至今}
\textit{在读硕士研究生(导师:马闪)}\ 应用数学, GPA: 3.32/4\\
兰州大学学业奖学金(3次), 预计 2020 年 7 月毕业
\datedsubsection{\textbf{淮北师范大学}, 淮北, 安徽}{2013 -- 2017}
\textit{学士}\ 应用数学, GPA: 3.1/4

\section{\faFlask\ 研究经历}
\subsection{\textbf{动力系统 - 发展方程解的长时间动力学行为}}
\begin{onehalfspacing}
\begin{itemize}
  \item 对一类退化的 $p$-Laplace 方程给出了吸引子的存在性,以便于进一步刻画解的长时间行为
  \item 得到了全局吸引子的正则性估计,以便于对吸引子的性质如维数等进行进一步的研究
  \item 主要框架是测度论、非线性泛函分析以及PDE先验估计等
\end{itemize}
\end{onehalfspacing}

\subsection{\textbf{机器学习 - 从动力系统的角度看待机器学习}}
\begin{onehalfspacing}
  \begin{itemize}
    \item 通过将隐层下标对应到时间把神经网络解释为离散的动力系统
    \item 从这个视角,可以通过PDE/ODE模型设计神经网络
    \item 同时可以将优化问题转化为神经网络的无监督模型
  \end{itemize}
\end{onehalfspacing}

\section{\faUsers\ 实习/项目经历}
\datedsubsection{\textbf{重离子加速器软件开发}}{2019年12月 -- 至今}
\role{Python, Qt5}{中科院项目,合作开发}
\begin{onehalfspacing}
用PyQt5开发重离子加速器软件界面以及后台逻辑
\begin{itemize}
  \item 负责分配工作以及写项目文档
  \item 用QWidget编写大部分界面,后期在考虑用QML重构
  \item 通过正则匹配相关变量存为json文件,再进一步自动生成部分逻辑代码
  \item 用PyQtGraph处理所有的数据绘图,按需添加数据缓存,数据的采样以及更新方式。
  \item 负责整体架构,将代码分成界面类、自动生成功能类和手动编写的功能类
\end{itemize}
\end{onehalfspacing}

% Reference Test
%\datedsubsection{\textbf{Paper Title\cite{zaharia2012resilient}}}{May. 2015}
%An xxx optimized for xxx\cite{verma2015large}
%\begin{itemize}
%  \item main contribution
%\end{itemize}

\section{\faCogs\ 相关技能}
% increase linespacing [parsep=0.5ex]
\begin{itemize}[parsep=0.5ex]
  \item 编程语言: C++, Python, Matlab
  \item 框架: Qt5, Pytorch, Tensorflow
  \item 平台: Linux, macOS
  \item 英语: CET6
\end{itemize}

\section{\faInfo\ 其他}
% increase linespacing [parsep=0.5ex]
\datedsubsection{无穷维动力系统及其吸引子国际会议}{2019.7}
\datedsubsection{兰州大学,\textit{暑期学校}}{2019.6}
\datedsubsection{厦门大学,\textit{暑期学校}}{2018.8}
\datedsubsection{非线性分析国际会议暨第二十届全国非线性泛函分析会议}{2018.5}

%% Reference
%\newpage
%\bibliographystyle{IEEETran}
%\bibliography{mycite}
\end{document}
