% !TEX TS-program = xelatex
% !TEX encoding = UTF-8 Unicode
% !Mode:: "TeX:UTF-8"

\documentclass{resume}
\usepackage{zh_CN-Adobefonts_external} % Simplified Chinese Support using external fonts (./fonts/zh_CN-Adobe/)
% \usepackage{NotoSansSC_external}
% \usepackage{NotoSerifCJKsc_external}
% \usepackage{zh_CN-Adobefonts_internal} % Simplified Chinese Support using system fonts
\usepackage{linespacing_fix} % disable extra space before next section
\usepackage{cite}
\usepackage{multicol}

\begin{document}
\pagenumbering{gobble} % suppress displaying page number

\name{李蕴方}

\basicInfo{
  \email{liyunfang@outlook.com} \textperiodcentered\ 
  \phone{(+86) 130-0877-2794} \textperiodcentered\ 
  % \github[@focus000]{https://github.com/focus000} \textperiodcentered\ 
  \linkedin[Yunfang-Li]{https://www.linkedin.com/in/yunfang-li-3ba402ba/}
  }
 
\section{\faGraduationCap\ 教育背景}
\datedsubsection{\textbf{兰州大学}, 兰州, 甘肃}{2017.9 -- 2020.6}
\textit{硕士}\ 应用数学, GPA: 3.32/4\\
兰州大学学业奖学金(3次)
\datedsubsection{\textbf{淮北师范大学}, 淮北, 安徽}{2013 -- 2017}
\textit{学士}\ 应用数学, GPA: 3.1/4\\
全国大学生数学建模竞赛三等奖

\section{\faBriefcase\ 工作经历}
\datedsubsection{\textbf{西安华为技术有限公司 C工程师}}{2020.8 -- 至今}
\textit{音视频会议SDK 音频模块开发}
\begin{itemize}
  \item 独立完成Windows端共享桌面时同时共享计算机声音特性开发与交付。代码量2k,转测缺陷密度<0.7个/kloc。
  技术上使用WASAPI实现电脑音频的采集与流路由,并对音频流做3A算法调优以及混音。
  \item 负责音频效果优化,使用隐马尔可夫模型自适应的识别多种场景,对算法内部做特定适配,解决了部分耳机场景下漏回声与过抑制的问题。
  通过对播放数据做自动增益控制以及自动调整播放与采集数据相对时延,解决扬声器模式下偶尔漏回声问题。
  通过对3A算法仿真与调参,大大改善了双讲下剪切的情况,最终大大提升了音效。在线上未出现音效类问题。
  \item 独立负责集成AI降噪与啸叫检测特性的设计开发交付,两块均作为明星特性,提升了产品竞争力。
  \item 作为音频模块问题接口人,参与定位音频问题30+,做到100\%解决。
  \item 在内部分享过3A算法设计与工作原理,总结算法代码架构以及多份疑难问题分析文档。
\end{itemize}
\textit{IdeaHub(会议大屏) 音频模块开发}
\begin{itemize}
  \item 参与现网音频断续问题定位,并给出规避方案,保障了大型多国会议的顺利召开。
  \item 在第三方会议软件在大屏上会议效果专项优化中,负责分析不同场景下第三方会议音效差问题,并给出各个场景下大屏音频配置文档
  \item 负责设计参考数据队列清理机制,解决了大屏上CPU高负荷运行时偶现漏回声的问题
\end{itemize}

% \section{\faFlask\ 研究经历}
% \subsection{\textbf{动力系统 - 发展方程解的长时间动力学行为}}
% \begin{onehalfspacing}
% \begin{itemize}
%   \item 对一类退化的 $p$-Laplace 方程给出了吸引子的存在性,以便于进一步刻画解的长时间行为
%   \item 得到了全局吸引子的正则性估计,以便于对吸引子的性质如维数等进行进一步的研究
%   \item 主要框架是测度论、非线性泛函分析以及PDE先验估计等
% \end{itemize}
% \end{onehalfspacing}

% \subsection{\textbf{机器学习 - 从动力系统的角度看待机器学习}}
% \begin{onehalfspacing}
%   \begin{itemize}
%     \item 通过将隐层下标对应到时间把神经网络解释为离散的动力系统
%     \item 通过PDE/ODE模型设计相应的网络
%     \item 将优化问题转化为神经网络的无监督模型
%   \end{itemize}
% \end{onehalfspacing}

% \section{\faUsers\ 实习/项目经历}
% \datedsubsection{\textbf{重离子加速器软件开发}}{2019.12 -- 2020.6}
% \role{Python, Qt5}{中科院项目,合作开发}
% \begin{onehalfspacing}
% 用Python + Qt5开发重离子加速器软件界面以及后台逻辑
% \begin{itemize}
%   \item 分配工作以及撰写项目文档以及软件测试
%   \item PyQtGraph大量数据绘图,QML做图形交互
%   \item 自定义QWidget控件
%   \item 多线程多进程编程
% \end{itemize}
% \end{onehalfspacing}

% Reference Test
%\datedsubsection{\textbf{Paper Title\cite{zaharia2012resilient}}}{May. 2015}
%An xxx optimized for xxx\cite{verma2015large}
%\begin{itemize}
%  \item main contribution
%\end{itemize}

\section{\faCogs\ 相关技能}
% increase linespacing [parsep=0.5ex]
% \vspace{-1\baselineskip}
% \begin{multicols}{1}
\begin{itemize}
    \item 编程语言: C(熟练), C++(了解), Python(熟练)
    \item 工具与框架: PyQt, Git, CMake, Windbg
    \item 英语: CET6
    \item 了解WebRTC设备管理源码实现与AEC3基本原理
    \item 了解VLC音频采集和播放源码实现
    \item 熟练使用 Visual Studio 和 XCode 进行 Debug 与性能分析
\end{itemize}
% \end{multicols}

% \section{\faInfo\ 其他}
% % increase linespacing [parsep=0.5ex]
% \datedsubsection{无穷维动力系统及其吸引子国际会议}{2019.7}
% % \datedsubsection{兰州大学,\textit{暑期学校}}{2019.6}
% \datedsubsection{厦门大学,\textit{暑期学校}}{2018.8}
% \datedsubsection{非线性分析国际会议暨第二十届全国非线性泛函分析会议}{2018.5}

\section{\faGamepad\ 关于我}
我正在寻找软件工程师职位,更倾向于后端以及底层软件开发,期待工作地点在长三角地区。
% 我是基础数学出身,但研究的内容无法落地,于是想做工程相关的工作,
% 期望在合适的时候可以结合自己的数学背景为公司和个人带来价值。
% 由于是零基础转方向,希望找到一份不常态化996的工作,因为需要业余时间努力提升自己。

%% Reference
%\newpage
%\bibliographystyle{IEEETran}
%\bibliography{mycite}
\end{document}
