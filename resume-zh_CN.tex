% !TEX TS-program = xelatex
% !TEX encoding = UTF-8 Unicode
% !Mode:: "TeX:UTF-8"

\documentclass{resume}
\usepackage{zh_CN-Adobefonts_external} % Simplified Chinese Support using external fonts (./fonts/zh_CN-Adobe/)
% \usepackage{NotoSansSC_external}
% \usepackage{NotoSerifCJKsc_external}
% \usepackage{zh_CN-Adobefonts_internal} % Simplified Chinese Support using system fonts
\usepackage{linespacing_fix} % disable extra space before next section
\usepackage{cite}

\begin{document}
\pagenumbering{gobble} % suppress displaying page number

\name{李蕴方}

\basicInfo{
  \email{liyunfang@outlook.com} \textperiodcentered\ 
  \phone{(+86) 137-0877-2794} \textperiodcentered\ 
  \github[@focus000]{https://github.com/focus000} \textperiodcentered\ 
  \linkedin[Yunfang-Li]{https://www.linkedin.com/in/yunfang-li-3ba402ba/}}
 
\section{\faGraduationCap\  教育背景}
\datedsubsection{\textbf{兰州大学}, 兰州, 甘肃}{2017 -- 至今}
\textit{在读硕士研究生(导师:马闪)}\ 应用数学, GPA: 3.32/4\\
兰州大学学业奖学金(3次), 预计 2020 年 7 月毕业
\datedsubsection{\textbf{淮北师范大学}, 淮北, 安徽}{2013 -- 2017}
\textit{学士}\ 应用数学, GPA: 3.1/4

\section{\faFlask\ 研究经历}
\subsection{\textbf{动力系统 - 发展方程解的长时间动力学行为}}
\begin{onehalfspacing}
本人硕士期间主要研究方向是无穷维动力系统及其吸引子,
旨在通过研究发展方程吸引子的性质,如存在性、正则性以及吸引子的维数等刻画解的长时间行为。
主要工具是PDE先验估计、非线性泛函分析与测度论。
本人在硕士期间主要工作是对一类加权$p$-Laplace抛物方程吸引子的存在性与正则性估计。
\end{onehalfspacing}

\subsection{\textbf{机器学习 - 神经ODE}}
鄂维南院士之前提出可以将神经网络看成连续动力系统,其中层看作时间点。
这样我们很多神经网络解释为离散的微分方程以及可以用处理随机微分方程的方法来优化网络结构。
这方面的理论化正由鄂维南教授团队在做,同时在计算机学界也有了很多应用、产生了有用的算法。
我曾尝试进一步将其应用在具体的网络上,如RNN,但效果并未有显著提高

\section{\faUsers\ 实习/项目经历}
\datedsubsection{\textbf{重离子加速器软件开发}}{2019年12月 -- 至今}
\role{Python, Qt5}{中科院项目,合作开发}
\begin{onehalfspacing}
用PyQt5开发重离子加速器软件界面以及后台逻辑
\begin{itemize}
  \item 负责整体架构,分配工作以及写项目文档
  \item 负责绘制大部分界面
  \item 自动生成大部分后台逻辑代码
  \item 负责所有的数据绘图,按需添加数据缓存
\end{itemize}
\end{onehalfspacing}

% Reference Test
%\datedsubsection{\textbf{Paper Title\cite{zaharia2012resilient}}}{May. 2015}
%An xxx optimized for xxx\cite{verma2015large}
%\begin{itemize}
%  \item main contribution
%\end{itemize}

\section{\faCogs\ 相关技能}
% increase linespacing [parsep=0.5ex]
\begin{itemize}[parsep=0.5ex]
  \item 编程语言: C, C++, Python, Matlab
  \item AI框架: Pytorch, Tensorflow
  \item 平台: Linux, macOS
  \item 开发: Qt5
  \item 英语: CET6
\end{itemize}

\section{\faInfo\ 其他}
% increase linespacing [parsep=0.5ex]
\datedsubsection{无穷维动力系统及其吸引子国际会议}{2019.7}
\datedsubsection{兰州大学,\textit{暑期学校}}{2019.6}
\datedsubsection{厦门大学,\textit{暑期学校}}{2018.8}
\datedsubsection{非线性分析国际会议暨第二十届全国非线性泛函分析会议}{2018.5}

%% Reference
%\newpage
%\bibliographystyle{IEEETran}
%\bibliography{mycite}
\end{document}
